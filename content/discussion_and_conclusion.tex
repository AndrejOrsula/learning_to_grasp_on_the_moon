\section{Discussion}\label{sec:discussion}

By training agents at two different levels of domain randomization, we investigated its effect on the sim-to-real transfer. However, the level of randomization significantly impacted the overall learning of agents. With fully-enabled randomization, agents encountered training instability that restricted their achievable success rate compared to agents that experienced a reduced subset of the lunar variety. We attribute the instability to high variance in gradient estimation due to large variations in observations. Despite this, sim-to-real experiments indicate that domain randomization enables robust transferability to the real-world domain, where agents trained in fully randomized environments performed better when transferred to the real robot despite their lower success rate in simulation. Although domain randomization enables learning of policies capable of generalization under challenging conditions, training stability must be significantly improved via methods such as active domain randomization~\cite{mehta_2019_active} in order to take full advantage of this approach.

The experimental evaluation indicates that 3D visual observations in the form of octrees provide better performance than image-based observations when applied for end-to-end learning of robotic grasping. This result is attributed to the better ability of 3D convolutions to generalize over spatial positions and orientations, unlike 2D convolutions that generalize well over planar image coordinates. Another advantage of 3D observations is their ability to provide learned policies with invariance to the camera pose, further simplifying the transfer to a different system or application domain. Sensory fusion of data from multiple depth sensors could also be applied to obtain a single 3D data structure that provides observations with improved quality and reduced occlusions. Therefore, these and other techniques must be investigated further to reveal the full potential of 3D observations in self-supervised learning of robot manipulation skills.

\section{Conclusion}\label{sec:conclusion}

In this work, we presented an approach for learning robotic grasping on the Moon with end-to-end deep reinforcement learning. We analyzed the application of 3D octree observations and compared their performance to 2D images. We also investigated the effects of employing domain randomization in lunar environments by demonstrating zero-shot sim-to-real transfer to a real robot in a Moon-analogue facility. Overall, we believe that deep reinforcement learning is a promising method for acquiring various manipulation skills for robots in space, despite its many challenges. Improving the learning stability in diverse environments is one of the necessary steps before similar approaches can be robustly employed for the wide range of applications within space robotics.
